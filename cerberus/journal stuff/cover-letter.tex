\documentclass{letter}

\usepackage[utf8]{inputenc}
\usepackage[T1]{fontenc}
\usepackage{hyperref}
\usepackage{mathrsfs}
\usepackage{xspace}
\newcommand\casper{{\sc Casper}\xspace}

\usepackage[]{xcolor}
\usepackage[]{pdfcomment}
\newcommand{\TODO}[1]{\textcolor{red}{todo: #1}\pdfcomment[color=yellow,open=false]{#1}}\newcommand\todo\TODO

%\signature{}

% default is today's date
%\date{Lille, March 24$^{th}$, 2016}

%% macros for writing response
% author: Vincenzo Musco
\newcounter{major}
\newcommand{\startmajorchanges}{{\Large\textbf{Major changes}}}
\newcommand{\majorchange}[1]{\refstepcounter{major}\item[M\arabic{major}]{#1}}
\newcounter{reviewercpt}
\newcounter{responsecpt}
\newcommand{\responsetonextreviewer}{\setcounter{responsecpt}{0}\refstepcounter{reviewercpt}{\Large\textbf{Response to reviewer \#\arabic{reviewercpt}}}}
\newcommand{\response}[2]{\refstepcounter{responsecpt}\item[R\arabic{reviewercpt}.\arabic{responsecpt}] "\textit{#1}"
    \\\textbf{Response: } #2}

\begin{document}

\begin{letter}{}
\opening{Dear P.Avgeriou and D. Shepherd,\\Editors-in-Chief, Journal of Systems and Software}

We would like to submit the attached manuscript, \emph{``ERRATUM: \em Leveraging~Flexible~Tree~Matching to~Repair~Broken~Locators in~Web~Automation~Scripts''}, for consideration for possible publication in \emph{``Journal of Systems and Software''}.
This manuscript (or closely related research) has not been published or accepted for publication. It is not under consideration at another journal.

\textsc{Abstract}:
Web applications are constantly evolving to integrate new features or fix reported bugs.
Such continuous evolutions may alter the content and/or the rendering of web application.
Underneath, all these web applications relies on the \emph{Document Object Model} (DOM) to render content from web browsers.
To interact with web applications, \emph{element locators} are identifiers used to navigate across the DOM when automating tasks (\emph{e.g.}. web test scripts, crawlers, or robotic process automation).
Unfortunately, element locators tend to become fragile when the underlying web pages evolve over time.
% 
\emph{Robust locators} have been introduced to overcome this issue, but they fail to repair broken locators on complex and dynamic web applications.
For this reason, several contributions explored the idea of automatically repairing broken locators on a page. 
However, these works attempt to repair a given broken locator by scanning all elements in the new DOM to find the most similar one, which fails to scale when the complexity of web pages grows.

This article adopts a different perspective on this problem by introducing a new locator repair solution that leverages tree matching algorithms to relocate broken locators.
This solution, named \erratum{}, implements a holistic approach to reduce the element search space, which greatly eases the locator repair task and drastically improves repair accuracy.
% 
We compare the robustness of \erratum{} on a large-scale benchmark composed of realistic and synthetic mutations applied to popular web applications currently deployed in production.
Our empirical results demonstrate that \erratum{} outperforms the accuracy of WATER, a state-of-the-art solution, by 67\%.

Should you require any further information, please contact us.

Yours sincerely,

Sacha Brisset\\
Mantu / Univ. Lille\\
\\
On behalf of Pr~Lionel Seinturier and Pr~Romain Rouvoy.
\end{letter}
\end{document}
