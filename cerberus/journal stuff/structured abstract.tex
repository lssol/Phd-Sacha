% Created 2021-02-22 Mon 16:56
% Intended LaTeX compiler: pdflatex
\documentclass[11pt]{article}
\usepackage[utf8]{inputenc}
\usepackage[T1]{fontenc}
\usepackage{graphicx}
\usepackage{grffile}
\usepackage{longtable}
\usepackage{wrapfig}
\usepackage{rotating}
\usepackage[normalem]{ulem}
\usepackage{amsmath}
\usepackage{textcomp}
\usepackage{amssymb}
\usepackage{capt-of}
\usepackage{hyperref}
\title{ERRATUM: Leveraging Flexible Tree Matching to Repair Broken Locators in Web Automation Scripts}
\begin{document}

\maketitle

\section{Context}
Web pages are constantly evolving to integrate new features and fix reported bugs.
Scripts that interact with web applications (\emph{e.g.} web test scripts, crawlers, or robotic process automation) rely on the underlying model of these web pages -- the Document Object Model (DOM) -- which means they are often particularly fragile.
More precisely, the major cause of breakages observed in automation scripts are \emph{element locators}, which are identifiers commonly used by automation scripts to navigate across the DOM.
When the DOM evolves, these identifiers tend to break, thus causing the related scripts to no longer locate the intended target elements.

For this reason, several contributions explored the idea of automatically repairing broken locators on a page.
These works attempt to repair a given broken locator by scanning all elements in the new DOM to find the most similar one.
This approach fails to scale when the complexity of web pages grows, leading to incorrect element repairs.
\section{Objective}
Our contribution aims at improving the robustness of scripts that interact with web applications.

In particular, we adopt a different perspective on the locator problem by introducing a new locator repair solution that leverages tree matching algorithms to relocate broken locators.

Finally, we propose a new way to interact with web pages enabled by ERRATUM.

\section{Method}
Our solution, named ERRATUM, implements a holistic approach to reduce the element search space, which greatly eases the locator repair task and drastically improves repair accuracy.

To evaluate ERRATUM, we build the first large-scale benchmark composed of realistic and synthetic mutations applied to popular web applications currently deployed in production.

\section{Results}
Our empirical results demonstrate that ERRATUM outperforms the accuracy of WATER, a state-of-the-art solution, by 67\%
\end{document}

    