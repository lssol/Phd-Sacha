\thispagestyle{empty}
\section*{Resumé}

\begin{doublespace}
Les applications Web sont omniprésentes dans la société moderne.
Certaines applications Web peuvent servir des millions de personnes.
Ces applications se doivent d'etre fiables et stables tout en étant capables d'évoluer pour s'adapter à ses utilisateurs.
À une telle échelle, ces attentes ne peuvent être satisfaites qu'avec d'importantes ressources.
Pour cette raison, il est essentiel d'approfondir notre capacité à comprendre la structure des applications Web pour faciliter leur maintenance et leur évolution.

Dans cette thèse, nous explorons la structure des applications Web à travers plusieurs perspectives : les tests Web, l'extraction de données et l'analyse Web.
Notre étude montre que de nombreuses recherches liées au Web, quel que soit le domaine de recherche, souffrent grandement de l'absence d'une solution générique d'inférence d'abstraction d'applications Web entièrement non supervisée. Nous tentons de développer une telle solution de manière itérative aboutissant à trois contributions principales :

\textbf{SFTM} Similarity-based Tree Matching, un algorithme permettant de faire correspondre deux pages web. Comparé aux algorithmes de correspondance d'arbres génériques traditionnels, SFTM produit de meilleures correspondances plus rapidement.

\textbf{ERRATUM} une approche permettant de réparer les localisateurs sur les applications web. ERRATUM améliore fortement la qualité des réparations pour peu ou pas de frais généraux. Nous avons intégré ERRATUM à logiciel de test open source largement utilisé.

\textbf{APPSTRACT} une approche pour générer automatiquement une abstraction d'une application web. APPSTRACT combine l'abstraction intra-page et l'abstraction inter-page à l'aide de SFTM pour générer des identifiants de localisation robustes et sémantiquement riches à l'échelle de l'application pour chaque élément d'une page Web.

Nous pensons que notre travail ouvre de nombreuses nouvelles possibilités dans une variété de domaines de recherche, en particulier : la vitesse de calcul de SFTM permet des approches qui n'étaient auparavant pas possibles avec l'appariement d'arbres génériques et l'approche que nous décrivons dans APPSTRACT pourrait ouvrir la voie à de nouvelles analyses Web ou à des solutions de génération de tests Web basées sur l'abstraction d'applications Web.
\end{doublespace}

